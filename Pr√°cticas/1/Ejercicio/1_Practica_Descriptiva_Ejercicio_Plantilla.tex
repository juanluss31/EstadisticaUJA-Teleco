% Options for packages loaded elsewhere
\PassOptionsToPackage{unicode}{hyperref}
\PassOptionsToPackage{hyphens}{url}
%
\documentclass[
]{article}
\usepackage{lmodern}
\usepackage{amssymb,amsmath}
\usepackage{ifxetex,ifluatex}
\ifnum 0\ifxetex 1\fi\ifluatex 1\fi=0 % if pdftex
  \usepackage[T1]{fontenc}
  \usepackage[utf8]{inputenc}
  \usepackage{textcomp} % provide euro and other symbols
\else % if luatex or xetex
  \usepackage{unicode-math}
  \defaultfontfeatures{Scale=MatchLowercase}
  \defaultfontfeatures[\rmfamily]{Ligatures=TeX,Scale=1}
\fi
% Use upquote if available, for straight quotes in verbatim environments
\IfFileExists{upquote.sty}{\usepackage{upquote}}{}
\IfFileExists{microtype.sty}{% use microtype if available
  \usepackage[]{microtype}
  \UseMicrotypeSet[protrusion]{basicmath} % disable protrusion for tt fonts
}{}
\makeatletter
\@ifundefined{KOMAClassName}{% if non-KOMA class
  \IfFileExists{parskip.sty}{%
    \usepackage{parskip}
  }{% else
    \setlength{\parindent}{0pt}
    \setlength{\parskip}{6pt plus 2pt minus 1pt}}
}{% if KOMA class
  \KOMAoptions{parskip=half}}
\makeatother
\usepackage{xcolor}
\IfFileExists{xurl.sty}{\usepackage{xurl}}{} % add URL line breaks if available
\IfFileExists{bookmark.sty}{\usepackage{bookmark}}{\usepackage{hyperref}}
\hypersetup{
  hidelinks,
  pdfcreator={LaTeX via pandoc}}
\urlstyle{same} % disable monospaced font for URLs
\usepackage[margin=1in]{geometry}
\usepackage{color}
\usepackage{fancyvrb}
\newcommand{\VerbBar}{|}
\newcommand{\VERB}{\Verb[commandchars=\\\{\}]}
\DefineVerbatimEnvironment{Highlighting}{Verbatim}{commandchars=\\\{\}}
% Add ',fontsize=\small' for more characters per line
\usepackage{framed}
\definecolor{shadecolor}{RGB}{248,248,248}
\newenvironment{Shaded}{\begin{snugshade}}{\end{snugshade}}
\newcommand{\AlertTok}[1]{\textcolor[rgb]{0.94,0.16,0.16}{#1}}
\newcommand{\AnnotationTok}[1]{\textcolor[rgb]{0.56,0.35,0.01}{\textbf{\textit{#1}}}}
\newcommand{\AttributeTok}[1]{\textcolor[rgb]{0.77,0.63,0.00}{#1}}
\newcommand{\BaseNTok}[1]{\textcolor[rgb]{0.00,0.00,0.81}{#1}}
\newcommand{\BuiltInTok}[1]{#1}
\newcommand{\CharTok}[1]{\textcolor[rgb]{0.31,0.60,0.02}{#1}}
\newcommand{\CommentTok}[1]{\textcolor[rgb]{0.56,0.35,0.01}{\textit{#1}}}
\newcommand{\CommentVarTok}[1]{\textcolor[rgb]{0.56,0.35,0.01}{\textbf{\textit{#1}}}}
\newcommand{\ConstantTok}[1]{\textcolor[rgb]{0.00,0.00,0.00}{#1}}
\newcommand{\ControlFlowTok}[1]{\textcolor[rgb]{0.13,0.29,0.53}{\textbf{#1}}}
\newcommand{\DataTypeTok}[1]{\textcolor[rgb]{0.13,0.29,0.53}{#1}}
\newcommand{\DecValTok}[1]{\textcolor[rgb]{0.00,0.00,0.81}{#1}}
\newcommand{\DocumentationTok}[1]{\textcolor[rgb]{0.56,0.35,0.01}{\textbf{\textit{#1}}}}
\newcommand{\ErrorTok}[1]{\textcolor[rgb]{0.64,0.00,0.00}{\textbf{#1}}}
\newcommand{\ExtensionTok}[1]{#1}
\newcommand{\FloatTok}[1]{\textcolor[rgb]{0.00,0.00,0.81}{#1}}
\newcommand{\FunctionTok}[1]{\textcolor[rgb]{0.00,0.00,0.00}{#1}}
\newcommand{\ImportTok}[1]{#1}
\newcommand{\InformationTok}[1]{\textcolor[rgb]{0.56,0.35,0.01}{\textbf{\textit{#1}}}}
\newcommand{\KeywordTok}[1]{\textcolor[rgb]{0.13,0.29,0.53}{\textbf{#1}}}
\newcommand{\NormalTok}[1]{#1}
\newcommand{\OperatorTok}[1]{\textcolor[rgb]{0.81,0.36,0.00}{\textbf{#1}}}
\newcommand{\OtherTok}[1]{\textcolor[rgb]{0.56,0.35,0.01}{#1}}
\newcommand{\PreprocessorTok}[1]{\textcolor[rgb]{0.56,0.35,0.01}{\textit{#1}}}
\newcommand{\RegionMarkerTok}[1]{#1}
\newcommand{\SpecialCharTok}[1]{\textcolor[rgb]{0.00,0.00,0.00}{#1}}
\newcommand{\SpecialStringTok}[1]{\textcolor[rgb]{0.31,0.60,0.02}{#1}}
\newcommand{\StringTok}[1]{\textcolor[rgb]{0.31,0.60,0.02}{#1}}
\newcommand{\VariableTok}[1]{\textcolor[rgb]{0.00,0.00,0.00}{#1}}
\newcommand{\VerbatimStringTok}[1]{\textcolor[rgb]{0.31,0.60,0.02}{#1}}
\newcommand{\WarningTok}[1]{\textcolor[rgb]{0.56,0.35,0.01}{\textbf{\textit{#1}}}}
\usepackage{graphicx,grffile}
\makeatletter
\def\maxwidth{\ifdim\Gin@nat@width>\linewidth\linewidth\else\Gin@nat@width\fi}
\def\maxheight{\ifdim\Gin@nat@height>\textheight\textheight\else\Gin@nat@height\fi}
\makeatother
% Scale images if necessary, so that they will not overflow the page
% margins by default, and it is still possible to overwrite the defaults
% using explicit options in \includegraphics[width, height, ...]{}
\setkeys{Gin}{width=\maxwidth,height=\maxheight,keepaspectratio}
% Set default figure placement to htbp
\makeatletter
\def\fps@figure{htbp}
\makeatother
\setlength{\emergencystretch}{3em} % prevent overfull lines
\providecommand{\tightlist}{%
  \setlength{\itemsep}{0pt}\setlength{\parskip}{0pt}}
\setcounter{secnumdepth}{-\maxdimen} % remove section numbering

\title{Prácticas de Estadística\\
Ejercicio de la Práctica 1\\
Estadística Descriptiva}
\author{true}
\date{25/11/2020}

\begin{document}
\maketitle

{
\setcounter{tocdepth}{2}
\tableofcontents
}
\hypertarget{presentaciuxf3n-del-ejercicio}{%
\section{Presentación del
ejercicio}\label{presentaciuxf3n-del-ejercicio}}

La hoja de datos \emph{airquality} de R contiene una serie de medidas
sobre variables que desciben la calidad del aire en Nueva York entre
mayo y septiembre de 1973. Concretamente, hay mediciones del nivel de
ozono (en ppb), de la radiación solar (en lang), de la velocidad
promedio del viento (en mph) y de la temperatura máxima diaria (en
grados Farenhait). Podemos ver todos los detalles sobre estos datos si
lanzamos a la consola de R \emph{?airquality}. Además, tenemos el día y
el mes en que se tomó cada medición.

El objetivo del ejercicio es realizar un análisis descriptivo de la
hoja.

En esta ocasión no es necesario importar la hoja porque ya pertenece al
entorno de trabajo que, por defecto, se incorpora al lanzar R.

Podemos ver un resumen inicial de todas las variables de esa hoja de
datos a continuación:

\begin{Shaded}
\begin{Highlighting}[]
\KeywordTok{summary}\NormalTok{(airquality)}
\end{Highlighting}
\end{Shaded}

\begin{verbatim}
##      Ozone           Solar.R           Wind             Temp      
##  Min.   :  1.00   Min.   :  7.0   Min.   : 1.700   Min.   :56.00  
##  1st Qu.: 18.00   1st Qu.:115.8   1st Qu.: 7.400   1st Qu.:72.00  
##  Median : 31.50   Median :205.0   Median : 9.700   Median :79.00  
##  Mean   : 42.13   Mean   :185.9   Mean   : 9.958   Mean   :77.88  
##  3rd Qu.: 63.25   3rd Qu.:258.8   3rd Qu.:11.500   3rd Qu.:85.00  
##  Max.   :168.00   Max.   :334.0   Max.   :20.700   Max.   :97.00  
##  NA's   :37       NA's   :7                                       
##      Month            Day      
##  Min.   :5.000   Min.   : 1.0  
##  1st Qu.:6.000   1st Qu.: 8.0  
##  Median :7.000   Median :16.0  
##  Mean   :6.993   Mean   :15.8  
##  3rd Qu.:8.000   3rd Qu.:23.0  
##  Max.   :9.000   Max.   :31.0  
## 
\end{verbatim}

\hypertarget{un-anuxe1lisis-estaduxedstico-descriptivo-sobre-indicadores-climuxe1ticos-del-aire-en-el-nueva-york-de-1973}{%
\section{Un análisis estadístico descriptivo sobre indicadores
climáticos del aire en el Nueva York de
1973}\label{un-anuxe1lisis-estaduxedstico-descriptivo-sobre-indicadores-climuxe1ticos-del-aire-en-el-nueva-york-de-1973}}

\hypertarget{resumen}{%
\subsection{Resumen}\label{resumen}}

El informe recoge las principales características, desde el punto de
vista descriptivo, de 4 indicadores relacionados con el clima y la
calidad del aire en Nueva York, medidos entre mayo y septiembre de 1973:
cantidad de ozono en el aire, nivel de radiación solar, velocidad del
viento y temperatura máxima.

\hypertarget{introducciuxf3n}{%
\subsection{Introducción}\label{introducciuxf3n}}

El clima es un fenómeno complejo sujeto a un alto nivel de incertidumbre
y observable a través de múltiples indicadores; algunas de estas
variables climáticas están relacionadas con la calidad del aire en las
ciudades, como el nivel de ozono.

El punto de partida de este análisis lo constituye una hoja de datos que
recopila 4 de esos indicadores en la ciudad de Nueva York a lo largo de
los meses de mayo, junio, julio, agosto y septiembre del año 1973. El
objetivo general del trabajo es el de proporcionar una visión global
sobre el clima en Nueva York y sobre la calidad de su aire en lo que
respecta al nivel de ozono. Como objetivos específicos, proponemos:

\begin{enumerate}
\def\labelenumi{\arabic{enumi}.}
\tightlist
\item
  Realizar un análisis descriptivo básico que incluya la descripción de
  la distribución de frecuencias, medidas de posición, dispersión y
  forma, de los cuatro indicadores que ofrece la hoja: nivel de ozono,
  radiación solar, velocidad del viento y temperatura máxima diaria.
\item
  Detectar qué días del año resultaron atípicos en relación con cada uno
  de los cuatro indicadores.
\end{enumerate}

\hypertarget{muxe9todologuxeda}{%
\subsection{Métodología}\label{muxe9todologuxeda}}

Como hemos comentado, la hoja recoge mediciones recogidas entre mayo y
septiembre de 1973, en la ciudad de Nueva York, de las siguientes
variables:

\begin{enumerate}
\def\labelenumi{\arabic{enumi}.}
\tightlist
\item
  Ozone: nivel medio de ozono en partes por billón desde las 13.00 hasta
  las 15.00 horas en la Isla de Roosevelt.
\item
  Solar.R: radiación solar en Langleys, en la frecuencia de banda de
  4000???7700 Angstroms desde las 08.00 hasta las 12.00 horas en Central
  Park.
\item
  Wind: velocidad promedio del viento en millas a la hora desde las 7.00
  hasta las 10.00 horas en el Aeropuerto de La Guardia.
\item
  Temp: temperatura máxima diaria en grados Fahrenheit en el aeropuerto
  de La Guardia.
\end{enumerate}

Además, dos variables indican el día y el mes de cada medición tomada.

En primer lugar, constataremos que se han recogido medidas todos los
días del período de observación. A continuación, para cada una de las
cuatro variables de interés, obtendremos:

\begin{enumerate}
\def\labelenumi{\arabic{enumi}.}
\tightlist
\item
  Una representación de su distribución de frecuencias. Dado que se
  trata de variables continuas, la herramienta será el histograma.
\item
  Medida de posición: media y cuartiles.
\item
  Medida de dispersión: coeficiente de variación.
\item
  Medida de forma: coeficiente de asimetría de Fisher.
\item
  Diagrama de caja donde puedan identificarse los días atípicos en
  cuanto al valor de cada variable.
\end{enumerate}

\hypertarget{resultados}{%
\subsection{Resultados}\label{resultados}}

En primer lugar, a modo de comprobación, obtenemos una tabla de
frecuencias absolutas de la variable \emph{mes} para comprobar que se
han tomado tantas medidas de cada variable como días tienen los meses
entre mayo y septiembre:

\begin{verbatim}
## 
##  5  6  7  8  9 
## 31 30 31 31 30
\end{verbatim}

Todo parece indicar que sí se tomaron medidas cada día de cada mes

Las 4 figuras que se muestran a continuación correponden a los
histogramas de las variables objeto del análisis.

\begin{Shaded}
\begin{Highlighting}[]
\KeywordTok{par}\NormalTok{(}\DataTypeTok{mfrow =} \KeywordTok{c}\NormalTok{(}\DecValTok{2}\NormalTok{, }\DecValTok{2}\NormalTok{))}
\KeywordTok{hist}\NormalTok{(airquality}\OperatorTok{$}\NormalTok{Ozone, }\DataTypeTok{ylab =} \StringTok{"Frecuencias"}\NormalTok{, }\DataTypeTok{xlab =} \StringTok{"Ozono, en ppb"}\NormalTok{, }\DataTypeTok{main =} \StringTok{""}\NormalTok{)}
\KeywordTok{hist}\NormalTok{(airquality}\OperatorTok{$}\NormalTok{Solar.R, }\DataTypeTok{ylab =} \StringTok{"Frecuencias"}\NormalTok{, }\DataTypeTok{xlab =} \StringTok{"Radiación solar, en lang"}\NormalTok{, }\DataTypeTok{main =} \StringTok{""}\NormalTok{)}
\KeywordTok{hist}\NormalTok{(airquality}\OperatorTok{$}\NormalTok{Wind, }\DataTypeTok{ylab =} \StringTok{"Frecuencias"}\NormalTok{, }\DataTypeTok{xlab =} \StringTok{"Velocidad del viento, en mph"}\NormalTok{, }\DataTypeTok{main =} \StringTok{""}\NormalTok{)}
\KeywordTok{hist}\NormalTok{(airquality}\OperatorTok{$}\NormalTok{Temp, }\DataTypeTok{ylab =} \StringTok{"Frecuencias"}\NormalTok{, }\DataTypeTok{xlab =} \StringTok{"Temperatura máxima diaria, en ºF"}\NormalTok{, }\DataTypeTok{main =} \StringTok{""}\NormalTok{)}
\end{Highlighting}
\end{Shaded}

\includegraphics{1_Practica_Descriptiva_Ejercicio_Plantilla_files/figure-latex/unnamed-chunk-3-1.pdf}

Por su parte, la tabla siguiente contiene las medidas descriptivas
mencionadas.

\begin{verbatim}
##                Ozono Radiación.solar Velocidad.del.viento Temperatura
## Media     42.1293103     185.9315068            9.9575163  77.8823529
## P25       18.0000000     115.7500000            7.4000000  72.0000000
## Me        31.5000000     205.0000000            9.7000000  79.0000000
## P75       63.2500000     258.7500000           11.5000000  85.0000000
## CV         0.7830151       0.4843634            0.3538032   0.1215329
## Coef.asim  1.2098656      -0.4192893            0.3410275  -0.3705073
\end{verbatim}

\hypertarget{anuxe1lisis}{%
\subsection{Análisis}\label{anuxe1lisis}}

Del análisis de las distribuciones de frecuencias podemos destacar los
siguientes aspectos:

\begin{itemize}
\tightlist
\item
  En el ozono la mayoría de los días hay niveles bajos, pero por alguna
  razón hay unos pocos días en que se dispara.
\item
  En la radiación solar parece pasar justo lo contrario: lo más
  frecuente son días con valores de radiación entre 200 y 300 lang, pero
  hay algunos días que destacan por su baja radiación. Quizá simplemente
  son días nublados.
\item
  En la velocidad del tiempo y la temperatura máxima diaria, sin
  embargo, parece que hay unos valores centrales más frecuentes y luego
  días de observaciones menores o mayores con las mismas frecuencias.
\end{itemize}

Estas valoraciones tienen que ver con la forma de las distribuciones de
frecuencias, que analizaremos de forma cuantitativa, mediante el
coeficiente de asimetría, a continuación.

\hypertarget{ozono}{%
\subsubsection{Ozono}\label{ozono}}

\href{https://airnow.gov/index.cfm?action=pubs.aqiguideozone}{En este
enlace} se considera que niveles de ozono por encima de 100 ppb son
peligrosos para grupos de riesgo. El nivel medio observado en nuestros
datos está por debajo de los 100, así como el percentil 75, lo que
indica que no más del 25 por ciento de los días se padecieron niveles
problemáticos de ozono. De hecho, el número de días por encima de 100
ppb fue de 7.

El coeficiente de variación refleja una dispersión moderada, es decir,
indica cierta variabilidad de los niveles de ozono a lo largo del
período de observación. Por su parte, el coeficiente de asimetría
confirma lo que observábamos en el histograma: una fuerte asimetría a la
derecha como consecuencia de la existencia de elevados niveles de ozono
en algunos días en particular.

\hypertarget{radiaciuxf3n-solar}{%
\subsubsection{Radiación solar}\label{radiaciuxf3n-solar}}

En \href{https://es.wikipedia.org/wiki/Langley}{Wikipedia} hemos
encontrado que:

\emph{La insolación anual en la parte alta de la atmósfera a diferentes
latitudes es: Para el polo la insolación anual es 133,2
kilolangleys/año. En el ecuador asciende a 320,9 kilolangleys/año, donde
el kilolangley=1000 langleys.}

Nuestros datos se refieren a un período de 4 horas. Tomemos como
referencia el ecuador, que recibe una radiación promedio en 4 horas de
219.7945205 Langleys. Por tanto, aunque la media de nuestros datos está
por debajo, no ocurre así con la mediana: ésta se aproxima bastante al
promedio por lo que podemos confirmar que al menos la mitad de los días
observados tuvieron una radiación por encima del promedio del ecuador.
La dispersión, según muestra el coeficiente de variación es
moderada-baja, indicando no muchas diferencias en la radiación a lo
largo de todo el período de observación. Finalmente, el coeficiente de
asimetría es sólo ligeramente negativo, indicando simetría de la
distribución de frecuencias: las diferencias observadas con respecto a
la radiación media se dan casi en el mismo sentido a la izquierda y a la
derecha de ésta.

\hypertarget{velocidad-del-viento}{%
\subsubsection{Velocidad del viento}\label{velocidad-del-viento}}

Vamos a valorar las medidas de posición en la
\href{https://en.wikipedia.org/wiki/Beaufort_scale}{escala de Beaufort}:

\begin{itemize}
\tightlist
\item
  La velocidad media del viento se considera brisa suave.
\item
  El percentil 25 se considera brisa ligera. Por tanto el 25\% de los
  días tuvieron vientos inferiores o iguales a brisa ligera.
\item
  La mediana es considerada brisa suave, así que el 50\% de los días
  hubo, como máximo, una brisa moderada.
\item
  El percentil 75 es considerado igualmente brisa suave. Por tanto, no
  más del 75\% de los días hubo vientos por encima de una brisa suave.
\end{itemize}

La cercanía entre los percentiles 75 y 25 ya indica no excesiva
variabilidad en la velocidad del viento, conclusión que se ve ratificada
por el coeficiente de variación, del 35.38\%. Finalmente, el coeficiente
de asimetría indica que la distribución de frecuencias es simétrica a
izquierda y derecha de la media.

\hypertarget{temperatura-muxe1xima-diaria}{%
\subsubsection{Temperatura máxima
diaria}\label{temperatura-muxe1xima-diaria}}

Para comparar las medidas descriptivas con algún valor de referencia,
vamos a considerar el promedio de los meses de mayo a septiembre en
Nueva York que aparecen
\href{https://www.currentresults.com/Weather/New-York/Places/new-york-city-temperatures-by-month-average.php}{en
este enlace}, que es de

\begin{Shaded}
\begin{Highlighting}[]
\NormalTok{(}\DecValTok{71}\OperatorTok{+}\DecValTok{79}\OperatorTok{+}\DecValTok{84}\OperatorTok{+}\DecValTok{83}\OperatorTok{+}\DecValTok{75}\NormalTok{)}\OperatorTok{/}\DecValTok{5}
\end{Highlighting}
\end{Shaded}

\begin{verbatim}
## [1] 78.4
\end{verbatim}

y se refieren al período 1981 a 2010:

\begin{itemize}
\tightlist
\item
  El valor medio de nuestros datos (corresponden a 1973) está
  ligeramente por debajo, pero no así la mediana.
\item
  Como cabe esperar, el percentil 25 está por debajo del valor medio
  entre 1981 y 2010, y el percentil 75 por encima.
\end{itemize}

Por otra parte, el coeficiente de variación es el menor de todos los
observados en las variables analizadas, e indica que el período de mayo
a septiembre de 1973 fue moderadamente estable en cuanto a las
temperaturas máximas diarias. Finalmente, el coeficiente de asimetría
también es cercano a cero, indicando simetría de las frecuencias de
valores a izquierda y derecha de la media.

\hypertarget{anuxe1lisis-de-la-presencia-de-datos-atuxedpicos}{%
\subsubsection{Análisis de la presencia de datos
atípicos}\label{anuxe1lisis-de-la-presencia-de-datos-atuxedpicos}}

Ahora vamos a identificar datos atípicos en cada una de las variables
observadas. En caso de existir datos atípicos, vamos a identificar el
día y el mes en que se dio.

\hypertarget{ozono-1}{%
\subsubsection{Ozono}\label{ozono-1}}

Mostramos a continuación el diagrama de caja con la identificación de
los valores atípicos:

\begin{Shaded}
\begin{Highlighting}[]
\NormalTok{datos <-}\StringTok{ }\NormalTok{airquality[}\OperatorTok{!}\KeywordTok{is.na}\NormalTok{(airquality}\OperatorTok{$}\NormalTok{Ozone), ]}\CommentTok{# Eliminamos los que no tienen valor observado}
\NormalTok{bp <-}\StringTok{ }\KeywordTok{boxplot}\NormalTok{(datos}\OperatorTok{$}\NormalTok{Ozone)}
\NormalTok{filtro <-}\StringTok{ }\NormalTok{datos}\OperatorTok{$}\NormalTok{Ozone }\OperatorTok{>}\StringTok{ }\NormalTok{bp}\OperatorTok{$}\NormalTok{stats[}\DecValTok{5}\NormalTok{] }\OperatorTok{|}\StringTok{ }\NormalTok{datos}\OperatorTok{$}\NormalTok{Ozone }\OperatorTok{<}\StringTok{ }\NormalTok{bp}\OperatorTok{$}\NormalTok{stats[}\DecValTok{1}\NormalTok{]}
\NormalTok{x <-}\StringTok{ }\KeywordTok{rep}\NormalTok{(}\FloatTok{1.25}\NormalTok{, }\KeywordTok{sum}\NormalTok{(filtro))}
\NormalTok{y <-}\StringTok{ }\NormalTok{datos}\OperatorTok{$}\NormalTok{Ozone[filtro]}
\NormalTok{id <-}\StringTok{ }\KeywordTok{paste}\NormalTok{(datos}\OperatorTok{$}\NormalTok{Day[filtro], }\StringTok{"th of month"}\NormalTok{, datos}\OperatorTok{$}\NormalTok{Month[filtro])}
\KeywordTok{text}\NormalTok{(x, y, id)}
\end{Highlighting}
\end{Shaded}

\includegraphics{1_Practica_Descriptiva_Ejercicio_Plantilla_files/figure-latex/unnamed-chunk-7-1.pdf}

\hypertarget{radiaciuxf3n-solar-1}{%
\subsubsection{Radiación solar}\label{radiaciuxf3n-solar-1}}

\begin{Shaded}
\begin{Highlighting}[]
\NormalTok{datos <-}\StringTok{ }\NormalTok{airquality[}\OperatorTok{!}\KeywordTok{is.na}\NormalTok{(airquality}\OperatorTok{$}\NormalTok{Solar.R), ]}\CommentTok{# Eliminamos los que no tienen valor observado}
\NormalTok{bp <-}\StringTok{ }\KeywordTok{boxplot}\NormalTok{(datos}\OperatorTok{$}\NormalTok{Solar.R)}
\end{Highlighting}
\end{Shaded}

\includegraphics{1_Practica_Descriptiva_Ejercicio_Plantilla_files/figure-latex/unnamed-chunk-8-1.pdf}

\hypertarget{velocidad-promedio-del-viento}{%
\subsubsection{Velocidad promedio del
viento}\label{velocidad-promedio-del-viento}}

\begin{Shaded}
\begin{Highlighting}[]
\NormalTok{datos <-}\StringTok{ }\NormalTok{airquality[}\OperatorTok{!}\KeywordTok{is.na}\NormalTok{(airquality}\OperatorTok{$}\NormalTok{Wind), ]}\CommentTok{# Eliminamos los que no tienen valor observado}
\NormalTok{bp <-}\StringTok{ }\KeywordTok{boxplot}\NormalTok{(datos}\OperatorTok{$}\NormalTok{Wind)}
\NormalTok{filtro <-}\StringTok{ }\NormalTok{datos}\OperatorTok{$}\NormalTok{Wind }\OperatorTok{>}\StringTok{ }\NormalTok{bp}\OperatorTok{$}\NormalTok{stats[}\DecValTok{5}\NormalTok{] }\OperatorTok{|}\StringTok{ }\NormalTok{datos}\OperatorTok{$}\NormalTok{Wind }\OperatorTok{<}\StringTok{ }\NormalTok{bp}\OperatorTok{$}\NormalTok{stats[}\DecValTok{1}\NormalTok{]}
\NormalTok{x <-}\StringTok{ }\KeywordTok{rep}\NormalTok{(}\FloatTok{1.25}\NormalTok{, }\KeywordTok{sum}\NormalTok{(filtro))}
\NormalTok{y <-}\StringTok{ }\NormalTok{datos}\OperatorTok{$}\NormalTok{Wind[filtro]}
\NormalTok{id <-}\StringTok{ }\KeywordTok{paste}\NormalTok{(datos}\OperatorTok{$}\NormalTok{Day[filtro], }\StringTok{"th of month"}\NormalTok{, datos}\OperatorTok{$}\NormalTok{Month[filtro])}
\KeywordTok{text}\NormalTok{(x, y, id)}
\end{Highlighting}
\end{Shaded}

\includegraphics{1_Practica_Descriptiva_Ejercicio_Plantilla_files/figure-latex/unnamed-chunk-9-1.pdf}

\hypertarget{temperatura-muxe1xima-diaria-1}{%
\subsubsection{Temperatura máxima
diaria}\label{temperatura-muxe1xima-diaria-1}}

\begin{Shaded}
\begin{Highlighting}[]
\NormalTok{datos <-}\StringTok{ }\NormalTok{airquality[}\OperatorTok{!}\KeywordTok{is.na}\NormalTok{(airquality}\OperatorTok{$}\NormalTok{Temp), ]}\CommentTok{# Eliminamos los que no tienen valor observado}
\NormalTok{bp <-}\StringTok{ }\KeywordTok{boxplot}\NormalTok{(datos}\OperatorTok{$}\NormalTok{Temp)}
\end{Highlighting}
\end{Shaded}

\includegraphics{1_Practica_Descriptiva_Ejercicio_Plantilla_files/figure-latex/unnamed-chunk-10-1.pdf}

\hypertarget{conclusiones}{%
\subsubsection{Conclusiones}\label{conclusiones}}

(Destaca las conclusiones más relevantes)

\end{document}
